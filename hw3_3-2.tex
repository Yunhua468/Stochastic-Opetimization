\documentclass{article}

\usepackage{amsmath,amsfonts,amssymb,amsthm}
\usepackage{mathtools}
\usepackage{comment}


\title{Exercise 3.2+4.4}
\author{Yunhua Zhao}
\date{\today}
\begin{document}
\maketitle

\textbf {3.2}  \\
a) Given initial value $\theta0$, recursively define the feedback process ${Y_n}$ through $$ \theta_{n+1} = \theta_n+\epsilon_nY_n $$
with either fixed step size $\epsilon$ or decreasing step size, where we typically assume that 
$$ \sum_{n=1}^{\infty}\epsilon_n = +\infty $$
$$ \sum_{n=1}^{\infty}\epsilon_n^2 < \infty $$
and $Y_n$ given via the feedback function
$$ Y_n = \phi(\xi(\theta_n),\theta_n) $$
 
We assume that all random variables, that is, $\theta0$ and $ ({\xi_n(\theta):n>=0, \theta\in\Theta}) $, are defined on a probability
space. Running the stochastic approximation algorithm, we observe the underlying
sequence
$$ \xi_0(\theta_0), \xi_1(\theta_1),... $$ 
Here in the problem, 
$$ \xi_1(\theta_1) = (0_{initial lose},(1-\theta_{0})_{initial A wins}, (-\theta_{0})_{initial B wins}) $$
$$ \xi_2(\theta_2) = (0_{1th lose},(1-\theta_{1})_{1th A wins}, (-\theta_{1})_{1th B wins}) $$
$$ \xi_3(\theta_3) = (0_{2th lose},(1-\theta_{2})_{2th A wins}, (-\theta_{2})_{2th B wins}) $$
$$ \xi_2(\theta_4) = (0_{3th lose},(1-\theta_{3})_{3th A wins}, (-\theta_{3})_{3th B wins}) $$
and so on, ...  \\

b) because $$ \theta_{n+1} = \theta_n+\epsilon_nY_n $$ 
set $Y_n(\xi_n(\theta_n))$ is the independent sequences of unbiased estimators of the target vector field, where
$$Y_n(\xi_n(\theta_n)) = (0_{n-1-th lose}, (1-\theta_{n-1})_{n-1-th A wins}, (-\theta_{n-1})_{n-1-th B wins}) $$


c) \textbf{Under} strict monotonicity, if choose A win, $Y_n=\xi_n(\theta_n) = 1-\theta_n$ the chosen direction the gradient is bigger than 0, which is always the grow direction;  \\
\textbf{And} the probability that B win,  $Y_n=\xi_n(\theta_n) = -\theta_n$ is always a descent direction, which is always the decent direction. \\
\textbf{So} this means that the field is coercive for the well-posed optimization problem.  \\

\begin{comment}
\textbf {4.4}  \\
Because $$J(\theta) = \frac{1}{2}E[(Z(X)-(\theta_1+\theta_2X))^2]) \ \ (1)$$
And $$ \theta_{n+1} = \theta_n-\epsilon_n\nabla_\theta J(\theta_n)^\intercal = \theta_n+\epsilon_nE[(Z(X)-\theta_{n,1}-\theta_{n,2}X)(1,X)^\intercal] \ \ (2) $$
From (2), we notice $$ Y_n = (\xi_n-\theta_n(1)-\theta_n(2)x_n)(1,x_n)^\intercal \ \ (3)$$
$Y_n$ satisfies the Martingale difference noise model  \\
The updates $Y_n$ are random variables.  \\
For each $x_n$ we obtain a corresponding random observation $ \xi_n=Z(x_n)$ \\
The underlying process is just a sequence of iid random pairs that are statistically independent of $\theta$. Algorithm (4.1) then becomes:
$$ \theta_{n+1,1} = \theta_{n,1}-\epsilon_n(\xi_n-\theta_{n,1}-\theta_{n,2}x_n) \ \ (4)$$
$$ \theta_{n+1,2} = \theta_{n,2}-\epsilon_n(\xi_n-\theta_{n,1}-\theta_{n,2}x_n) \ \ (5)$$
From the question, 
$$ \nabla J(\theta) = (-E[Z(X)-\theta_1-\theta_2X], -E[XZ(X)-\theta_1X-\theta_2X^2])^\intercal  \ \ (6) $$
So
$$ -\nabla J(\theta) = (E[Z(X)-\theta_1-\theta_2X], E[XZ(X)-\theta_1X-\theta_2X^2])^\intercal  \ \ (7) $$

The difference between the trend and the realization is denoted by 
$$ \delta M_n = Y_n-E[Y_n|\mathfrak{F_{n-1}}] $$
Admits a Taylor approximation to $E[Y_n|\mathfrak{F_{n-1}}]$ 
$$ E[Y_n|\mathfrak{F_{n-1}}] = \frac{J(\theta_n+c_n)-J(\theta_n)}{2c_n} - \frac{J(\theta_n-c_n)-J(\theta_n)}{2c_n} $$
According to example 4.1 $$ = $$
\end{comment}



\textbf {Mohamed4.4}  \\
Show that  for a random variable x with finite variance
$$ \nabla J(\theta) = (-E[Z(X)-\theta_1-\theta_2X], -E[XZ(X)-\theta_1X-\theta_2X^2])^\intercal  \ \ (1) $$
$$J(\theta) = \frac{1}{2}E[(Z(X)-(\theta_1+\theta_2X))^2] \ \ (2)$$
Which we could get:
$$\frac{\partial J(\theta)}{\partial \theta_1} = -E[Z(X)-(\theta_1-\theta_2X)] \ \ (3) $$
$$\frac{\partial J(\theta)}{\partial \theta_2} = -E[XZ(X)-\theta_1X-\theta_2X^2]  \ \ (4)$$
For each $x_n$ we obtain a corresponding random observation $ \xi_n=Z(x_n)$ \\
$$ E(Z(x_n)) = h(x_n) $$
The feedback function is 
 $$ Y_n = (\xi_n-\theta_n(1)-\theta_n(2)x_n)(1,x_n)^\intercal \ \ (5)$$
 
Because $x_n$ and $Z(x_n)$ are random, so $Y_n$ is independent:
\begin{flushleft}
$ E[Y_n|\mathfrak{F_{n-1}}] = E[(\xi_n-\theta_n(1)-\theta_n(2)x_n)(1,x_n)^\intercal] $  \\
$ \ \ \ \ \ \ \ \ \ \ \ \ \ \ \ \ =E[(Z(x_n)-\theta_n(1)-\theta_n(2)x_n, x_nZ(x_n)-\theta_n(1)x_n-\theta_n(2)x_n^2)^\intercal] $  \\
$ \ \ \ \ \ \ \ \ \ \ \ \ \ \ \ \ =(E[Z(x_n)-\theta_n(1)-\theta_n(2)x_n], E[x_nZ(x_n)-\theta_n(1)x_n-\theta_n(2)x_n^2])^\intercal $  \\
$ \ \ \ \ \ \ \ \ \ \ \ \ \ \ \ \ =-\nabla J(\theta_n(1),\theta_n(2)) $  \\
$ \ \ \ \ \ \ \ \ \ \ \ \ \ \ \ \ =-\nabla J(\theta_n) $
\end{flushleft} 


\textbf {Mohamed4.3}  \\
\begin{flushleft}
$E[M_{n+1}|\mathfrak{F}_n] $ \\
$= E[(M_n+\epsilon_{n+1} \delta M_{n+1})|\mathfrak{F}_n]$  \\
$= E[M_n|\mathfrak{F}_n]+\epsilon_{n+1}E[\delta M_{n+1}|\mathfrak{F}_n]]$  \\
$= E[M_n|\mathfrak{F}_n]+\epsilon_{n+1}E[(y_{n+1}-E[y_{n+1}|\mathfrak{F}_n])|\mathfrak{F}_n] $  \\
$= E[M_n|\mathfrak{F}_n]+\epsilon_{n+1}E[y_{n+1}|\mathfrak{F}_n]-\epsilon_{n+1}E[E[(y_{n+1}|\mathfrak{F}_n)|\mathfrak{F}_n]] $ \\

Because $$\mathfrak{F}_{n-1} \in \mathfrak{F}_n$$

$=E[M_n|\mathfrak{F}_{n-1}]+\epsilon_{n+1}E[y_{n+1}|\mathfrak{F}_n]-\epsilon_{n+1}E[y_{n+1}|\mathfrak{F}_n]$ \\
Here: 
$$ E[M_n|\mathfrak{F}_{n-1}] = M_n $$
And
$$ \epsilon_{n+1}E[y_{n+1}|\mathfrak{F}_n]-\epsilon_{n+1}E[y_{n+1}|\mathfrak{F}_n] = 0 $$
So:
$$ E[M_{n+1}|\mathfrak{F}_n] = M_n $$

















\end{flushleft} 




































































\end{document}