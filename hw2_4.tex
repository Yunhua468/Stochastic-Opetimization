\documentclass{article}

\usepackage{amsmath,amsfonts,amssymb,amsthm}
\usepackage{mathtools}


\title{Exercise 2.4}
\author{Yunhua Zhao}
\date{\today}
\begin{document}
\maketitle
(a) because $$x_{n}(t)=\vartheta^{\epsilon}(t_{n}+t)$$ and $$\vartheta^{\epsilon}(t)=\theta_{m(t)}$$ 
so $$x_{n}(t)=\theta_{m(t+t_n)}$$ 
   $$x_{n}(t+s)=\theta_{m(t+s+t_n)} $$
then $$x_{n}(t+s)-x_{n}(t)=\theta_{m(t+s+t_n)}-\theta_{m(t+t_n)}$$
which $$ =\sum_{i=m(t_n+t)}^{m(t_n+t+s)-1}\epsilon_iG(\theta_i) $$
Because $X_\epsilon(.)$ is piecewise point, $G(X_\epsilon(.))$ is also piecewise constant and its jump times are given by $t_n=\sum_{k=1}^{n}\epsilon_k$. Thus the definite integral on $[t_n+t, t_n+t+s]$ of $G(X_\epsilon(.))$ is a sum that can be approximation expressed as $$ \int_{t_n+t}^{t_n+t+s}G[x_\epsilon(u)]du $$ 
together  $$ \int_{t_n+t}^{t_n+t+s}G[x_\epsilon(u)]du  \approx \sum_{i=m(t_n+t)}^{m(t_n+t+s)-1}\epsilon_iG(\theta_i) $$ 
$$ \int_{t_n+t}^{t_n+t+s}G[x_\epsilon(u)]du=\sum_{i=m(t_n+t)}^{m(t_n+t+s)-1}\epsilon_iG(\theta_i)+\rho_(\epsilon),                    (2.1)$$ \\


(b) formula $$x_n(t+s)-x_n(t)=\theta_{m(t+s+t_n)}-\theta_{m(t+t_n)}=\sum_{i=m(t_n+t)}^{m(t_n+t+s)-1}\epsilon_iG(\theta_i) $$
contains $m(q)-m(r)-1$ terms. For $\epsilon$ sufficiently small, set the $\epsilon_b$ is the biggest $\epsilon$ and the $\epsilon_s$ is the smallest $\epsilon$ in interval (r,q) %$m(r)=m(t_n+t)>=\frac{r}{\epsilon}$ and $m(q)=m(t_n+t+s)=<\frac{q}{\epsilon}$, 
so that the number of terms is bounded by ($\frac{q-r}{\epsilon_b}$, $\frac{q-r}{\epsilon_s}$). This yields, for small $\epsilon$,
$$  \lVert(x_\epsilon(q)-x_\epsilon(r))\rVert_\infty = \sum_{i=m(r)}^{m(q)-1}\epsilon_iG(\theta_i) $$
Because G is bounded, let use L to represent G's bounder, so 
$$  \lVert x_\epsilon(q)-x_\epsilon(r) \rVert_\infty = L\sum_{i=m(r)}^{m(q)-1}\epsilon_i =<\epsilon_b L(q-r)/\epsilon_s,                               (2.2)$$

To summarize, for $\epsilon$ sufficiently small, we have shown that for any $\eta>0$, we may let $\delta_\eta=\frac{\eta}{L(\epsilon_b/\epsilon_s)}/$ so that it follows that $\lVert x_\epsilon(q)-x_\epsilon(r) \rVert_\infty =<\eta$ wherever $\lVert q-r \rVert =< \delta_\eta(\epsilon_b/\epsilon_s)$.  
This establishes equicontinuity in the extended sense.\\


(c) Let $a<t$ and $b>t+s$ and consider $x_{\epsilon_k}(.)$ on (a,b). Set $x_n(0)=\theta_0$ for all k. Therefor, for $\epsilon$ sufficiently small, by (b) formula 2.2,
$$ {\lvert x_{\epsilon_k}(r) \rvert}_\infty =< {\lvert \theta_0 \rvert}_\infty + rL\frac{\epsilon_b}{\epsilon_s} $$
for all $r>0$, which suffices to show that $x_\epsilon$ is uniformly bounder in (a,b). This together with equicontinuity of $x_{\epsilon_k}$ implies by the Ascoli-Arzela Theorem 2.2 that any infinite subsequence of $x_{\epsilon_k}$ has a convergent subsequence with a continuous limit on (a, b). Consider a convergent subsequence along $\epsilon_r \rightarrow 0$, so that $\hat{x}(.)=\lim_{\epsilon_r \rightarrow 0}x_{\epsilon_r}(.)$ (in the sup norm)  and continuous. Then 
$$ \lim_{\epsilon_r \rightarrow 0}(x_{\epsilon_r}(t+s)-x_{\epsilon_r}(t))=\lim_{\epsilon_r \rightarrow 0}\int_{t}^{t+s}G(x_{\epsilon_r}(u))du $$
$$ =\int_{t}^{t+s}\lim_{\epsilon_r \rightarrow 0}G(x_{\epsilon_r}(u))du $$
$$ =\int_{t}^{t+s}G(\hat{x}(u))du $$
Where the first formula follows from the fact that $\rho(\epsilon)$ in (2.1) is bounded by $L(\epsilon_b+\epsilon_s)$ and thus of order $\mathcal{O}(\epsilon)$, the second formula follows from Lebesgue Dominated Convergence Theorem, and the third formula is a consequence
of the continuity of $G(\hat{x}(.))$ on (a,b). We arrive for $s>0$ at
$$ \frac{\hat{x}(t+s)-\hat{x}(t)}{s}=\frac{1}{s}\int_{t}^{t+s}G(\hat{x}(u))du $$
By continuity of $G(\hat{x}(.))$, taking the limit as s goes to zero, the above right-hand side converges to $G(\hat{x}(t))$, which establishes the ODE in the question for $\hat{x}(.)$. Because G is continuous and bounded on the trajectory $\hat{x}$,  it follows from Theorem 2.1 that the ODE has a unique solution for each initial condition,establishing that all accumulation points have the same limit, proving the claim for the unbiased
case. \\


(d) if the bias is not 0, then first:
$$ \int_{t_n+t}^{t_n+t+s}G[x_\epsilon(u)]du=\sum_{i=m(t_n+t)}^{m(t_n+t+s)-1}\epsilon_iG(\theta_i)+\sum_{i=m(t_n+t)}^{m(t_n+t+s)-1}\epsilon_i\beta_n(\theta_i)+\rho_(\epsilon) $$
using the bound on the perturbations, for any $r<q$, 
$$ \sum_{i=m(t_n+t)}^{m(t_n+t+s)-1}\epsilon_i\beta_n(\theta_i)=<\sum_{i=m(t_n+t)}^{m(t_n+t+s)-1}\epsilon_b\beta_n(\theta_i)=(q-r)\mathcal{O}(\epsilon_b) $$
so this term can be added to the approximation error $\rho_(\epsilon)$ and the proof follows directly.

























\end{document}